%%%%%%%%%%%%%%%%%%%%%%%%%%%%%%%%%%%%%%%%%%%%%%%%%%%%%%%%%%%%%%%%%%%%%%%%%%%%%%%%
%2345678901234567890123456789012345678901234567890123456789012345678901234567890
%        1         2         3         4         5         6         7         8
% THESIS INTRODUCTION

\chapter{Introduction}
\label{chap:introduction}
\ifpdf
    \graphicspath{{Introduction/Figures/PNG/}{Introduction/Figures/PDF/}{Introduction/Figures/}}
\else
    \graphicspath{{Introduction/Figures/EPS/}{Introduction/Figures/}}
\fi

% quote

%\setlength{\epigraphwidth}{.35\textwidth}
%\epigraph{Research is formalized curiosity.}{ Zora Neale Hurston, 1942}

% examples of sections

% \section{Motivations}
% \label{motivations}
  
% \section{Context of the Study}
% \label{context}

\section{Project overview}
\label{introduction}

Music information retrieval (MIR) is an area of analysis dedicated to extracting
information from music. It combines many different disciplines of science
including psychology, psychoacoustics, signal processing and computer science.
One of the main aims when applying MIR techniques solving the task of song
identification, i.e. matching an audio stream to a particular song
\cite{weinstein2007music}. This is usually achieved through a form of hashing
applied on the digital signal and comparing the resulting representation to a
reference fingerprint \cite{wang2003industrial}, \cite{haitsma2001robust}. This
approach returns good results for the task, since we can easily quantify a good
match between both fingerprints.

We can further modify the original song identification task to apply to cover
songs. A cover song is a very creative reinterpretation of a released song
usually performed by an artist different than the original. The cover can
therefore differ significantly from the origin in tempo, pitch or song structure
(\textit{add more}). The amount of variation in a cover strongly depends on the
genre of the primary track - Western popular music pieces are for example more
likely to be transformed than ones from classical music
\cite{ellis2007identifyingcover}. Therefore the only remaining common feature
between the cover song and the original is the underlying fundamental melody of
the piece and potentially the lyrics.

Because of these potential disparities between two versions of a single song,
the problem of identifying covers of songs is much more difficult than determing
an identical match with the original. The above fingerprinting approach
has been attempted \cite{bertin2011large} and the results are insignificant
\cite{ellis2012large}. Direct comparison between the fingerprints of the song is
unable to capture the remaining similarity within two audio files. Other MIR
methods need to be considered in order to measure similarity when attempting
cover song recognition.

The general advances of technology have allowed companies such as Spotify
\cite{spotify}, Apple \cite{applemusic}, SoundCloud \cite{soundcloud} and more
to create large-scale music databases and offer them as commercial services.
Proportionally to the increasing availability of large music collections grows
the need for managing the volumes of audio information through MIR techniques,
with cover song identification being one of them. As a consequence most modern
mechanisms to cover song recognition work by comparing an audio track called
\textit{query song} against a large database of songs, a \textit{reference
database}. Each mechanism is evaluated based on its similarity estimation
performance, as well as its scalability as we increase the database size.

This project analyses the principles of a set of non-hashing based cover song
identification algorithms and evaluates their performance. Most of the examined
algorithms are designed to work with large-scale databases and follow the
workflow model described above. The evaluation considers only their similarity
estimation results and does not account for scalability. After analysis of the
results a hypothesis on the best performing audio similarity technique is
established (\textit{or maybe devised?}).

\section{Report structure}
\label{structure}
The sections of this report are as follows:
\begin{enumerate}
    \item Chapter 2 offers a summary of the background information required to
    understand and implement the audio similarity algorithms
    \item Chapter 3 explores other state of the art methods of measuring
    similarity not examined in detail by the project
    \item Chapter 4 provides a description of the evaluation task through which
    each algorithm is analysed
    \item Chapter 5 contains detailed descriptions of each algorithm
    \item Chapter 6 expands on implementation details related to the benchmark
    tool
    \item Chapter 7 outlines the best results achieved and offers an analysis on
    them
    \item Chapter 8 discusses the main challenges related to the project and the
    task of cover song identification
    \item Chapter 9 is a summary of the project management techniques utilised
    during the project
    \item Chapter 10 presents 

\end{enumerate}