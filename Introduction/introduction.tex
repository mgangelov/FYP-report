%%%%%%%%%%%%%%%%%%%%%%%%%%%%%%%%%%%%%%%%%%%%%%%%%%%%%%%%%%%%%%%%%%%%%%%%%%%%%%%%
%2345678901234567890123456789012345678901234567890123456789012345678901234567890
%        1         2         3         4         5         6         7         8
% THESIS INTRODUCTION

\chapter{Introduction}
\label{chap:introduction}
\ifpdf
    \graphicspath{{Introduction/Figures/PNG/}{Introduction/Figures/PDF/}{Introduction/Figures/}}
\else
    \graphicspath{{Introduction/Figures/EPS/}{Introduction/Figures/}}
\fi

% quote

%\setlength{\epigraphwidth}{.35\textwidth}
%\epigraph{Research is formalized curiosity.}{ Zora Neale Hurston, 1942}

% examples of sections

% \section{Motivations}
% \label{motivations}
  
% \section{Context of the Study}
% \label{context}

\section{Introduction}
\label{introduction}

Music information retrieval (MIR) is an area of analysis dedicated to extracting information from music. It combines many different disciplines of science including psychology, psychoacoustics, signal processing and computer science. One of the main aims when performing MIR solving the task of song identification, i.e. matching an audio stream to a particular song \nocite{weinstein2007music}


\section{Report structure}
\label{structure}
