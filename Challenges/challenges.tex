%%%%%%%%%%%%%%%%%%%%%%%%%%%%%%%%%%%%%%%%%%%%%%%%%%%%%%%%%%%%%%%%%%%%%%%%%%%%%%%%
%2345678901234567890123456789012345678901234567890123456789012345678901234567890
%        1         2         3         4         5         6         7         8
% THESIS CHAPTER

\chapter{Challenges}
\label{chap:challenges}
\ifpdf
    \graphicspath{{Algorithms/Figures/PNG/}{EvaluationTask/Figures/PDF/}{Algorithms/Figures/}}
\else
    \graphicspath{{Algorithms/Figures/EPS/}{EvaluationTask/Figures/}}
\fi


% short summary of the chapter
Throughout the time frame of the project several challenges were encountered and
overcome to the best of my abilities. Some of them heavily influenced the
algorithms analysis, implementation and achieved results. This section describes
these obstacles. 

\section{Lack of supporting academic infrastructure} 
\label{sec:academicinfra}

The task of cover song identification is still a relatively new one within the
MIR area. Active contributions to it are significantly less than other
admittedly more fundamental and important problems such as chord estimation or
beat tracking \cite{mirex}. 

Because of this relative lack of innovation, the task suffers from shortage of
good evaluation criteria. The only metrics used to examine the performance of an
algorithm are the ones used in the \textit{Audio Cover Song Identification} part
of the MIREX conference (and consequently this project). Their current
definition however fails to establish a reliable method of comparing results
from different datasets, as discussed in Chapter \ref{chap:results}.

The amount of big publicly available datasets is also insufficient. Due to
copyright concerns most datasets decide to provide extracted audio features from
a set of songs rather the songs themselves. While it has been established which
features perform best in relation to the task (eg. chroma features), the lack of
the original audio streams applies an inherent limitation of the flexibility and
originality of a future research project. The biggest dataset of features from
cover songs is the \textit{SecondHandSongs (SHS)} dataset \cite{shs}. It is
sufficiently big in size (18,196 tracks in total), however it is not easily
obtainable as it is distributed as part of the \textit{Million Song Dataset} - a
dataset consisting of 1,000,000 songs and having a size of around 300 GB.
Because of its large magnitude the dataset is available only to researchers
using cloud-based virtual machines to interact with it. Apart from it the only
other publicly available datasets include \textit{covers80} (used in this
project) and \textit{covers1000} (offering a small number of audio features
extracted from 1000 songs). 

\section{Project scope}
\label{sec:projectscope}

The general project scope was initially underestimated by a margin, possibly due
to my inexperience in the area of digital signal processing (DSP) prior to
starting the project. The overall increasing complexity of it was unexpected and
the required theoretical background and development skills presented many
challenges. Almost all of them were overcome through thorough research into
related concepts and techniques. Consequently this analysis on the broader DSP
field contributed to the depth of knowledge the project aims to achieve.

Despite my best efforts some concepts still present a challenge for me to
understand.

\section{Ambiguity in the algorithm papers} 
\label{sec:papersambiguity}

Implementing all algorithms was another significant hurdle to prevail over. The
main purpose of each paper describing a new way of identifying cover or live
songs is to inform about the introduced innovations and the results achieved as
a consequence. A lot of implementation details are omitted which makes the task
of reproducing the algorithm one that requires a lot of experimentation and
improvisation.

Some academics have shared their algorithm codebase online - this
proved to be very helpful in filling the empty spaces of the algorithm
specification defined in the published paper.

